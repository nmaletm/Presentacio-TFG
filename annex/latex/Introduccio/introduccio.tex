\section{Introducció}

Actualment la Universitat Politècnica de Catalunya (UPC) disposa d'una gran quantitat de plataformes per recolzar la docència i la gestió de la universitat. La majoria d'aquestes eines estan dissenyades per assistir als docents i estudiants mitjançant un conjunt d'aplicacions per facilitar la gestió documental, per oferir un canal de comunicació entre el docent i l'estudiant (Atenea o el Racó de la FIB) o simplement un servei molt concret (com pot ser el LEARN-SQL que ofereix un entorn per fer docència de base de dades).

UPCnet\cite{upcnet}, que és el client d'aquest projecte (el projecte SomUPC), ha detectat que encara així, hi ha mancances en l'aspecte comunitari. Aquest mètodes no són els ideals per comunicar-se amb la gent jove, i es vol dissenyar una nova plataforma que eviti la dependència al correu electrònic, que tingui les xarxes socials més presents i, que permeti interacció entre els membres de la comunitat.

Per aquest motiu, ha dissenyat un sistema en el que hi ha un motor d'activitat social i una sèrie d'aplicacions que s'alimenten d'aquest i el nodreixen. Aquest motor d'activitat social rep el nom de MAX i d'ara en endavant l'anomenarem així. A l'annex es pot trobar una explicació més detallada del MAX (secció \ref{sec:max}).

Aquestes aplicacions poden ser sistemes ja existents, com és el cas d'Atenea, o noves eines com són el portal SomUPC i les aplicacions natives per a dispositius mòbils iOS i Android.

En concret, aquest treball final de grau es concentra en el desenvolupament de l'aplicació iOS per al MAX, que és una de les peces del SomUPC.

UPCnet vol desenvolupar aquesta eina per a la UPC, però també vol que es pugui implantar en qualsevol altra comunitat que ho pugui necessitar, per exemple una empresa. Per tant, el sistema ha de ser molt flexible per a que s'adapti el millor possible a qualsevol organització.