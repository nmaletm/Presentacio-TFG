
\section{Conclusions}

Vaig començar aquest projecte sense saber que acabaria sent el meu Treball Final de Grau, durant la meva estada com a becari al inLab FIB. Aproximadament a mig projecte l'Albert hem va oferir l'oportunitat de presentar-ho com a treball i després de fer tots els tràmits amb una setmana em vaig trobar immers a l'assignatura de GEP (Gestió de Projectes).

Abans de l'inici d'aquest projecte havia desenvolupat unes quantes aplicacions natives d'iOS, però mai m'havia trobat amb un projecte d'aquesta envergadura i abast. Estic molt content ja que m'ha ofert la possibilitat de veure créixer una aplicació des de zero i veure com organitzar l'estructura interna d'aquesta. Estic segur que es podria haver fet molt millor, però estic content del resultat obtingut.

Un altre aspecte que m'ha agradat molt d'aquest projecte, és que alhora que creixia l'aplicació iOS també ho feia l'aplicació Android. Gracies a que les dues evolucionaven a un ritme igual, els membres de l'equip, hem pogut comparar com afronten els mateixos problemes els dos sistemes. Ens hem adonat com, tot hi que puguin semblar iguals, els dos sistemes tenen plantejaments molt diferents i a vegades difícil fins i tot de comparar. En algunes ocasions hem pogut solucionar problemes d'un sistema aplicar patrons de l'altre sistema.

Al projecte SomUPC hem aplicat la metodologia \textit{Scrum} i crec que ha estat una de les claus que han facilitat l'èxit del projecte. Al iniciar el projecte, teníem dues grans peces, el portal SomUPC i les aplicacions natives clients del MAX.

Al inici es va prioritzar més el portal, però es va donar un punt en que les prioritats del client van canviar i es va prioritzar completament les aplicacions. Inclús durant un període de temps, es paralitzar el portal fins acabar les aplicacions. 

Aquest canvi, sota el meu punt de vista, era necessari fer-ho, ja que el client per motius de negoci necessitava les aplicacions. Si no hagués estat per la flexibilitat de la metodologia potser no hauria estat possible. Si no s'hagués fet, el client segurament no s'hauria pogut queixar ja que el pacte inicial no contemplava el canvi. Però el fet de ser flexibles i adaptar-nos al client ha fet que aquest estigui content amb el projecte i li pugui donar continuïtat.

En aquest projecte hem treballat conjuntament l'equip SomUPC i UPCnet. Hem realitzat moltes reunions, l'experiència obtinguda en aquestes reunions és la major aportació que m'emporto d'aquest projecte. Estic segur que això m'aportarà molt al meu futur professional. Quan acabi la carrera i m'enfronti a la primera reunió de treball agrairé haver realitzat aquest projecte. Encara recordo la primera reunió que vam fer, i els nervis que vaig passar abans de començar.

Per acabar, m'agradaria agrair a tots els membres que han col·laborat en aquest projecte, tot l'equip SomUPC i l'equip d'UPCnet (negoci i equip tècnic).
