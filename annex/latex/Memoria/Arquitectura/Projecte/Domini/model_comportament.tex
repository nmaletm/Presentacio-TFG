
\subsubsection{Model de comportament}

En aquesta memòria, s'ha inclòs un model de comportament per cada un dels quatre tipus d'operacions que té el sistema, seguint amb la tria que s'ha fet a la secció d'especificació del sistema. 

El model conceptual segueix el patró de disseny \textit{Domain Model}, ja que es delega a la classe responsable de cada acció la responsabilitat de realitzar-la. 

La classe \textit{UPCMaxController} és la responsable de fer les crides al servidor. Per tant, utilitza la llibreria \textit{RestKit}. Aquesta llibreria funciona amb crides asíncrones mitjançant l'us de delegats (el patró delegat està explicat a la secció \ref{sec:patro_delegat}), la llibreria delega al \textit{UPCMaxController} la tasca de gestionar la resposta del servidor.

El \textit{UPCMaxController} quan rep una resposta del servidor envia una notificació (el patró notificació està explicat a la secció \ref{sec:patro_notificacio}) del tipus pertinent, per exemple \textit{timeline} actualitzat amb èxit o error al carregar el \textit{timeline}. I les vistes que s'han registrat com a observadores de la notificació, la reben i realitzen els canvis pertinents a la vista.


% Al paint té una alçada de 2369

\begin{figure}[ht]
    \centering
    \includegraphics*[scale=0.6, viewport=0 1800 800 2369]{Memoria/Arquitectura/Projecte/Domini/model_comportament_arquitectura.png}
    \caption{Diagrama de seqüència d'una operació d'obtenció d'informació.}
    \label{fig:model_comportament_obtenir}
\end{figure}
\FloatBarrier


\begin{figure}[ht]
    \centering
    \includegraphics*[scale=0.5, viewport=0 1120 906 1800]{Memoria/Arquitectura/Projecte/Domini/model_comportament_arquitectura.png}
    \caption{Diagrama de seqüència d'una operació de publicació d'informació.}
    \label{fig:model_comportament_publicar}
\end{figure}
\FloatBarrier


\begin{figure}[ht]
    \centering
    \includegraphics*[scale=0.6, viewport=0 570 800 1120]{Memoria/Arquitectura/Projecte/Domini/model_comportament_arquitectura.png}
    \caption{Diagrama de seqüència d'una operació d'esborrat.}
    \label{fig:model_comportament_esborrar}
\end{figure}
\FloatBarrier


\begin{figure}[ht]
    \centering
    \includegraphics*[scale=0.6, viewport=0 0 800 570]{Memoria/Arquitectura/Projecte/Domini/model_comportament_arquitectura.png}
    \caption{Diagrama de seqüència d'una operació de modificació de dades.}
    \label{fig:model_comportament_modificacio}
\end{figure}
\FloatBarrier
