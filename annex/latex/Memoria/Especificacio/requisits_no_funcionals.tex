\subsection{Requisits no funcionals}

A més dels requisits funcionals de l'apartat anterior, el sistema ha de complir una serie de requisits no funcionals. Aquests requisits estan definits seguint una plantilla simplificada que inclou la següent informació:

\begin{compactitem}
    \item Identificador
    \item Nom del requeriment
    \item Descripció del requeriment
\end{compactitem}

A continuació es pot veure primer un exemple del format, i després tots els requisits no funcionals del sistema.


\definecolor{colorCell}{rgb}{0.78,0.85,0.94}
\definecolor{colorBorder}{rgb}{0.39,0.64,0.68}

% Primer definim les macros que pintaran els requisits
\newcommand{\pintarRequisit}[2]{
    \pintarRequisitBox{\arabic{numeroRequisit}}{#1}{#2}
    \stepcounter{numeroRequisit}
}

\newcommand{\pintarRequisitBox}[3]{
    \begin{table}[ht]
        \setlength{\arrayrulewidth}{0.5mm}
        \arrayrulecolor{colorBorder}
        \begin{tabularx}{\linewidth}{ | l | l | }
           \hline
            \cellcolor{colorCell}\textit{\textbf{ReqNF. \#:}} & #1 \\ \hline
            \multicolumn{2}{|X|}{\cellcolor{colorCell}\textit{\textbf{Nom:}}} \\ \hline
            \multicolumn{2}{|X|}{#2} \\ \hline
            \multicolumn{2}{|X|}{\cellcolor{colorCell}\textit{\textbf{Descripció:}}} \\ \hline
            \multicolumn{2}{|X|}{#3} \\ \hline
        \end{tabularx}
    \end{table}
    \FloatBarrier
    \arrayrulecolor{colorBorderDefecte}
}



% Preparem el contador de requisits a 1
\newcounter{numeroRequisit}\stepcounter{numeroRequisit}
\def\arraystretch{1.5}

\bgroup


\pintarRequisitBox
    {Identificador}
    {Nom del requeriment}
    {Descripció i justificació del requeriment.}

\pintarRequisit
    {Usable}
    {El sistema ha de ser senzill d'utilitzar i intuïtiu per als usuaris als que va dirigit.}
    
\pintarRequisit
    {Multi-idioma}
    {El sistema ha d'estar disponible en Català, Espanyol i Anglès. I ha de ser fàcil de traduir a altres idiomes.}

\pintarRequisit
    {Robust}
    {El sistema no ha de permetre l'entrada per part de l'usuari de dades incorrectes. Per tal de mantenir la integritat del sistema, també ha de garantir que no es produeixin duplicitats.}

\pintarRequisit
    {Complir la \textit{App Review Guidelines}}
    {El producte s'ha d'adaptar a la guia de revisió d'aplicacions d'\textit{Apple} (\textit{App Review Guidelines}  \cite{apple_app_review_guideliness}).}
    
\pintarRequisit
    {Canviable}
    {El sistema ha de permetre realitzar canvis al sistema de manera que aquests no afectin a les funcionalitats ja existents sense requerir molt temps per a realitzar-los.}

\pintarRequisit
    {Disseny UPC}
    {L'aplicació ha de tenir un disseny d'acord amb el disseny institucional de la UPC, i per tant ha de seguir les guies d'estil de la UPC\cite{guia_estil_upc}.}
    
\pintarRequisit
    {Extensible}
    {El sistema ha de ser fàcilment extensible. Ha de ser possible afegir fàcilment noves funcionalitats així com millorar les funcionalitats existents.}
    
\pintarRequisit
    {Compatible última versió iOS}
    {El sistema ha de ser com a mínim compatible amb la última versió del sistema operatiu iOS. Intentant mantenir el màxim de compatibilitat amb versions anteriors del sistema operatiu.}

\egroup

\clearpage