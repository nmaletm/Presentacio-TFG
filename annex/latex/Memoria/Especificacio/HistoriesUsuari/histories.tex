\subsubsection{Històries d'usuari}
\label{sec:histories_usuari}

% http://www.genbetadev.com/metodologias-de-programacion/historias-de-usuario-una-forma-natural-de-analisis-funcional

La metodologia \textit{scrum} utilitza les històries d'usuari per definir les diferents funcionalitats del sistema. Aquestes històries d'usuari estan definides pel client i posades en comú amb tot l'equip als \textit{sprint plannings}.

Cada historia d'usuari esta composada per un títol que les identifica, el text de la història, un número d'història d'usuari (que no implica cap ordre ni prioritat), una xifra de prioritat i els criteris d'acceptació.

Els criteris d'acceptació son els criteris que ha de complir la funcionalitat a desenvolupar, per a que negoci accepti la funcionalitat com a bona.


% Primer definim la macro que pintarà les histories

\definecolor{colorCellHistories}{rgb}{0.78,0.85,0.94}
\definecolor{colorBorderHistories}{rgb}{0.39,0.64,0.68}

\newcommand{\pintaHistoria}[4]{

    \label{sec:historia_\arabic{numeroHistoriaUsuari}}

    \begin{table}[ht]
        \setlength{\arrayrulewidth}{0.5mm}
        \arrayrulecolor{colorBorderHistories}
        \begin{tabularx}{\linewidth}{|l|l|}
            \hline
            \multicolumn{2}{|X|}{\large{\textbf{\arabic{numeroHistoriaUsuari}}. \textbf{#1}}} \\ \hline
            \multicolumn{2}{|X|}{#2} \\ \hline
            \cellcolor{colorCellHistories}\textbf{Prioritat:} & #3 \\ \hline
            \multicolumn{2}{|X|}{\cellcolor{colorCellHistories}\textbf{Criteris d'acceptació:}} \\ \hline
            \multicolumn{2}{|X|}{\begin{minipage}[t]{\linewidth}#4\end{minipage}} \\ \hline
            
        \end{tabularx}
    \end{table}

    \arrayrulecolor{colorBorderDefecte}

    \stepcounter{numeroHistoriaUsuari}
    \FloatBarrier
}


% Preparem el comptador d'històries a 1
\newcounter{numeroHistoriaUsuari}\stepcounter{numeroHistoriaUsuari}

\bgroup
\def\arraystretch{1.5}

\pintaHistoria
    {Iniciar sessió}
    {\textbf{Com a} usuari \textbf{jo vull} poder iniciar sessió amb el meu usuari \textbf{per a poder} accedir a la meva informació personal.}
    {80}
    {
    \begin{itemize}[leftmargin=0.3cm]
        \item Vull emplenar un formulari amb les meves credencials per iniciar sessió.
        \item No he de poder entrar amb unes credencials invalides.
    \end{itemize}
    }

\pintaHistoria
    {Veure \textit{timeline}}
    {\textbf{Com a} usuari registrat \textbf{jo vull} veure les activitats del meu \textit{timeline} \textbf{per a poder} veure les activitats publicades als espais que estic subscrit.}
    {75}
    {
    \begin{itemize}[leftmargin=0.3cm]
        \item Vull veure totes les activitats del meu \textit{timeline}.
        \item Per cada activitat vull veure el començament del contingut, l'autor (el nom i la seva imatge), la data de publicació i el número de comentaris.
        \item Vull poder clicar a sobre per poder veure el detall de l'activitat.
    \end{itemize}
    }

\pintaHistoria
    {Veure activitat}
    {\textbf{Com a} usuari registrat \textbf{jo vull} accedir a la informació d'una activitat \textbf{per a poder} veure el contingut que s'ha publicat, l'autor que ho ha fet i el moment en que ho ha fet.}
    {60}
    {
    \begin{itemize}[leftmargin=0.3cm]
        \item Vull veure el contingut complet de l'activitat.
        \item Vull veure el nom i la imatge del seu autor.
        \item Vull veure la data de publicació de l'activitat.
    \end{itemize}
    }

    
\pintaHistoria
    {Veure comentaris d'una activitat}
    {\textbf{Com a} usuari registrat \textbf{jo vull} poder veure els comentaris d'una activitat \textbf{per a poder} saber la opinió o els comentaris que han publicat els altres usuaris de l'activitat.}
    {60}
    {
    \begin{itemize}[leftmargin=0.3cm]
        \item Vull veure tots els comentaris que s'han publicat a una activitat.
        \item De cada comentari vull veure el contingut del comentari, l'autor (nom i imatge) i la data de publicació del comentari.
    \end{itemize}
    }
    
\pintaHistoria
    {Crear una activitat}
    {\textbf{Com a} usuari registrat \textbf{jo vull} publicar una activitat a un espai al que tinc permisos d'escriptura \textbf{per a poder} compartir una informació amb els altres usuaris.}
    {65}
    {
    \begin{itemize}[leftmargin=0.3cm]
        \item Vull escriure a un camp el contingut de l'activitat.
        \item Vull seleccionar dels contexts als que puc escriure a quin/quins publicar l'activitat.
        \item No he de poder publicar a un context que no tinc permís d'escriptura.
        \item Quan guardi, s'ha de publicar l'activitat. 
    \end{itemize}
    }
    
\pintaHistoria
    {Afegir un comentari}
    {\textbf{Com a} usuari registrat \textbf{jo vull} afegir un comentari a una activitat \textbf{per a poder} fer saber als altres usuaris una opinió o comentari.}
    {50}
    {
    \begin{itemize}[leftmargin=0.3cm]
        \item Vull escriure a un camp el meu comentari.
        \item Quan guardi s'ha de publicar el comentari a l'activitat.
    \end{itemize}
    }

\pintaHistoria
    {Veure subscripcions}
    {\textbf{Com a} usuari registrat \textbf{jo vull} veure totes les meves subscripcions \textbf{per a poder} revisar els espais als que estic subscrit.}
    {70}
    {
    \begin{itemize}[leftmargin=0.3cm]
        \item Vull veure totes les meves subscripcions agrupades per tipologia (Assignatures, Institucionals, Comunitats o altres).
        \item Si selecciono un context, vull veure les activitas d'aquest context.
    \end{itemize}
    }
    
\pintaHistoria
    {Veure activitats d'una subscripció}
    {\textbf{Com a} usuari registrat \textbf{jo vull} veure les activitats d'un espai al que estic subscrit \textbf{per a poder} veure totes les activitats que han generat els usuaris de l'espai.}
    {65}
    {
    \begin{itemize}[leftmargin=0.3cm]
        \item Vull veure totes les activitats publicades a un context.
        \item Per cada activitat vull veure el començament del contingut, l'autor (el nom i la seva imatge), la data de publicació i el número de comentaris.
        \item Vull poder clicar a sobre per poder veure el detall de l'activitat.
    \end{itemize}
    }

\pintaHistoria
    {Afegir una subscripció}
    {\textbf{Com a} usuari registrat \textbf{jo vull} afegir noves subscripcions \textbf{per a poder} veure les activitats publicades a un espai del que ara no estic subscrit.}
    {65}
    {
    \begin{itemize}[leftmargin=0.3cm]
        \item Vull afegir una nova subscripció a un context.
        \item No he de poder subscriurem a un context que no sigui públic.
        \item Se m'ha de confirmar si realment vull subscriurem al context.
        \item Si accepto, s'ha de crear la subscripció al context.
        \item Si cancel·lo, no s'ha de crear la subscripció.
    \end{itemize}
    }

\pintaHistoria
    {Cercar contexts públics}
    {\textbf{Com a} usuari registrat \textbf{jo vull} cercar un context públic \textbf{per a poder} subscriurem al context.}
    {65}
    {
    \begin{itemize}[leftmargin=0.3cm]
        \item Vull tenir un camp a on introduir el nom del context a cercar.
        \item He de veure els contexts que compleixen la cerca.
    \end{itemize}
    }

\pintaHistoria
    {Veure perfil usuari}
    {\textbf{Com a} usuari registrat \textbf{jo vull} veure el meu perfil d'usuari \textbf{per a poder} veure quines dades meves té el sistema, i revisar que les dades publicades són correctes.}
    {70}
    {
    \begin{itemize}[leftmargin=0.3cm]
        \item Vull veure el meu nom.
        \item Vull veure el meu nom d'usuari.
        \item Vull veure el meu \textit{twitter}.
        \item Vull veure la meva imatge.
    \end{itemize}
    }

\pintaHistoria
    {Veure activitats d'un usuari}
    {\textbf{Com a} usuari registrat \textbf{jo vull} veure les activitats d'un usuari \textbf{per a poder} veure les activitats que aquest ha generat al sistema.}
    {50}
    {
    \begin{itemize}[leftmargin=0.3cm]
        \item Vull veure totes les activitats que he publicat.
        \item Per cada activitat vull veure el començament del contingut, l'autor (el meu nom i la meva imatge), la data de publicació i el número de comentaris.
        \item Vull poder clicar a sobre per poder veure el detall de l'activitat.
    \end{itemize}
    }

\pintaHistoria
    {Modificar dades perfil usuari}
    {\textbf{Com a} usuari registrat \textbf{jo vull} modificar les meves dades d'usuari \textbf{per a poder} actualitzar la informació que està publicada al sistema.}
    {60}
    {
    \begin{itemize}[leftmargin=0.3cm]
        \item Vull tenir un camp per modificar el meu nom, i que inicialment posi el meu nom antic.
        \item Vull tenir un camp per modificar el meu \textit{twitter}, i que inicialment posi el meu \textit{twitter} antic.
        \item Si guardo, vull que el meu perfil es modifiqui amb les noves dades.
    \end{itemize}
    }
    
\pintaHistoria
    {Veure converses usuari}
    {\textbf{Com a} usuari registrat \textbf{jo vull} veure totes les converses a les que participo \textbf{per a poder} triar una conversa i veure els missatges de la conversa.}
    {80}
    {
    \begin{itemize}[leftmargin=0.3cm]
        \item Vull veure totes les meves converses. 
        \item Per cada conversa vull veure el començament del últim missatge, el nom del grup, la data del últim missatge, el número de missatges i la imatge de la conversa (si és una conversa de dos, la imatge de l'altre participant, si és de més una imatge fixe).
        \item Vull poder clicar a sobre per poder veure els missatges de la conversa.
    \end{itemize}
    }

\pintaHistoria
    {Veure missatges d'una conversa}
    {\textbf{Com a} usuari registrat \textbf{jo vull} veure els missatges d'una conversa \textbf{per a poder} fer el seguiment de la conversa, llegint els missatges nous o els missatges ja llegits.}
    {80}
    {
    \begin{itemize}[leftmargin=0.3cm]
        \item Vull veure tots els missatges de la conversa.
        \item Els missatges els vull veure en format de xat.
        \item Per a cada missatge vull veure el text del missatge i la data de quan es va enviar.
        \item Per als missatges que no sigui meus, vull veure el nom de l'autor del missatge.
    \end{itemize}
    }

\pintaHistoria
    {Afegir un nou missatge a una conversa}
    {\textbf{Com a} usuari registrat \textbf{jo vull} afegir un nou missatge a una conversa \textbf{per a poder} comunicar un missatge als altres usuaris de la conversa.}
    {70}
    {
    \begin{itemize}[leftmargin=0.3cm]
        \item Vull tenir un camp a on poder posar el text d'un nou missatge.
        \item Si no he introduït cap missatge, no he de poder enviar.
        \item Quan envií el missatge, aquest s'ha d'enviar a la resta d'usuaris.
        \item Quan envií el missatge, el botó d'enviar s'ha de desactivar i s'ha de buidar el camp de text.
    \end{itemize}
    }

\pintaHistoria
    {Visualitzar dades d'una conversa}
    {\textbf{Com a} usuari registrat \textbf{jo vull} visualitzar les dades d'una conversa \textbf{per a poder} veure el nom del grup i els membres que hi participen.}
    {60}
    {
    \begin{itemize}[leftmargin=0.3cm]
        \item Vull veure el nom de la conversa.
        \item Vull veure el nom i foto de tots els participants de la conversa.
        \item A la llista vull veure destacat el creador de la conversa.
    \end{itemize}
    }

\pintaHistoria
    {Modificar dades d'una conversa}
    {\textbf{Com a} usuari registrat propietari de la conversa \textbf{jo vull} modificar les dades de la conversa \textbf{per a poder} canviar el nom del grup i afegir o treure usuaris de la conversa.}
    {50}
    {
    \begin{itemize}[leftmargin=0.3cm]
        \item Vull tenir un camp per modificar el nom de la conversa, amb el nom antic inicialment.
        \item Vull poder afegir nous participants a la conversa.
        \item Vull poder treure participants de la conversa.
        \item Se m'ha de confirmar abans de treure o posar una persona a la conversa.
        \item Si accepto, s'ha de afegir/treure a la persona.
        \item Si cancel·lo, no s'ha de afegir/treure a la persona.
        \item No he de poder modificar la conversa si no soc el creador d'aquesta.
    \end{itemize}
    }

\pintaHistoria
    {Crear una conversa}
    {\textbf{Com a} usuari registrat \textbf{jo vull} iniciar una nova conversa \textbf{per a poder} iniciar una conversa amb un conjunt d'usuaris.}
    {60}
    {
    \begin{itemize}[leftmargin=0.3cm]
        \item Vull tenir un camp per afegir els participants.
        \item Vull tenir un camp per introduir el missatge.
        \item Si creo la conversa i hi han més de dos participants, se m'ha de preguntar el nom del grup.
    \end{itemize}
    }
    
\pintaHistoria
    {Notificacions de converses}
    {\textbf{Com a} usuari registrat \textbf{jo vull} rebre notificacions al dispositiu mòbil dels nous missatges \textbf{per a poder} assabentar-me que un altre usuari ha enviat un missatge a una conversa en la que participo.}
    {50}
    {
    \begin{itemize}[leftmargin=0.3cm]
        \item Quan algun usuari hagi enviat un missatge a una conversa meva, vull que se'm notifiqui d'aquest missatge.
        \item No vull rebre notificacions quan l'aplicació estigui oberta.
    \end{itemize}
    }
    
\pintaHistoria
    {Converses en temps real}
    {\textbf{Com a} usuari registrat \textbf{jo vull} seguir les converses en temps real \textbf{per a poder} mantenir una conversa fluida amb una persona o un grup de persones.}
    {50}
    {
    \begin{itemize}[leftmargin=0.3cm]
        \item Quan tingui l'aplicació oberta vull rebre els missatges de les meves converses en temps real.
        \item Vull que em surti el número de converses que tenen missatges nous a la barra de pestanyes.
        \item Vull que a la llista de converses surtin marcades les converses amb missatges nous.
        \item Si tinc una conversa oberta, vull que quan un altre usuari publiqui un missatge, a mi em surti en temps real a la llista de missatges.
    \end{itemize}
    }

\pintaHistoria
    {Esborrar una activitat}
    {\textbf{Com a} usuari registrat \textbf{jo vull} esborrar una activitat que he publicat \textbf{per a poder} evitar que els altres usuaris vegin una activitat que no vull que puguin veure.}
    {35}
    {
    \begin{itemize}[leftmargin=0.3cm]
        \item Vull esborrar una activitat lliscant la cel·la de l'activitat cap a un costat.
        \item Abans d'esborrar-ho, vull que se'm demani confirmació de l'eliminació.
        \item Si accepto, s'ha d'esborrar l'activitat.
        \item Si cancel·lo, no s'ha d'esborrar l'activitat.
    \end{itemize}
    }
    
\pintaHistoria
    {Esborrar un comentari}
    {\textbf{Com a} usuari registrat \textbf{jo vull} esborrar un comentari meu o un comentari d'una activitat que he publicat jo \textbf{per a poder} evitar que els usuaris vegin un comentari que no vull que puguin veure.}
    {30}
    {
    \begin{itemize}[leftmargin=0.3cm]
        \item Vull esborrar un comentari lliscant la cel·la del comentari cap a un costat.
        \item Abans d'esborrar-ho, vull que se'm demani confirmació de l'eliminació.
        \item Si accepto, s'ha d'esborrar el comentari.
        \item Si cancel·lo, no s'ha d'esborrar el comentari.
    \end{itemize}
    }
    
\pintaHistoria
    {Esborrar una subscripció}
    {\textbf{Com a} usuari registrat \textbf{jo vull} esborrar una subscripció \textbf{per a poder} deixar de veure les seves activitats al meu \textit{timeline}.}
    {40}
    {
    \begin{itemize}[leftmargin=0.3cm]
        \item Vull esborrar una subscripció lliscant la cel·la de la subscripció cap a un costat.
        \item Abans d'esborrar-ho, vull que se'm demani confirmació de l'eliminació.
        \item Si accepto, s'ha d'esborrar la subscripció.
        \item Si cancel·lo, no s'ha d'esborrar la subscripció.
    \end{itemize}
    }
   
\pintaHistoria
    {Sortir/Esborrar una conversa}
    {\textbf{Com a} usuari registrat \textbf{jo vull} esborrar/sortir una conversa \textbf{per a poder} deixar de rebre els missatges d'una conversa que no vull seguri més.}
    {40} 
    {
    \begin{itemize}[leftmargin=0.3cm]
        \item Vull esborrar/sortir una conversa lliscant la cel·la de la conversa cap a un costat.
        \item Abans d'esborrar-ho, vull que se'm demani confirmació de l'eliminació.
        \item Si accepto, s'ha d'esborrar/sortir de la conversa.
        \item Si cancel·lo, no s'ha d'esborrar/sortir de la conversa.
        \item Només puc esborrar una conversa si sóc el propietari.
        \item Només puc sortir d'una conversa si no sóc el propietari.
    \end{itemize}
    }

\pintaHistoria
    {Tancar sessió}
    {\textbf{Com a} usuari registrat \textbf{jo vull} tancar la sessió \textbf{per a poder} canviar d'usuari.}
    {40} 
    {
    \begin{itemize}[leftmargin=0.3cm]
        \item Vull tenir un botó al perfil d'usuari per tancar la sessió.
        \item Abans de tancar la sessió se m'ha de demanar una confirmació.
        \item Si accepto, s'ha de tancar la meva sessió a l'aplicació.
        \item Si cancel·lo, no s'ha de tancar la meva sessió a l'aplicació.
    \end{itemize}
    }
    
\pintaHistoria
    {Publicar un arxiu a una activitat}
    {\textbf{Com a} usuari registrat \textbf{jo vull} afegir una imatge o un fitxer a una activitat \textbf{per a poder} compartir-la amb els altres usuaris de l'espai.}
    {50}
    {
    \begin{itemize}[leftmargin=0.3cm]
        \item Vull tenir un botó per afegir una imatge (de la galeria o de la càmera) a l'activitat que estic publicant.
        \item Vull que des de les altres aplicacions pugui obrir fitxers amb aquesta aplicació, i quan ho faci, que se'm posi com a fitxer adjunt.
        \item No he de poder publicar una activitat amb un fitxer al meu context personal.
        \item No he de poder publicar una activitat amb un fitxer a un context que no és una comunitat.
    \end{itemize}
    }
    
\egroup

\clearpage

% Forma antiga de pintar-ho
    
\newcommand{\pintaHistoriaImatge}[3]{
    \begin{tikzpicture}
        \draw (0, 0) node[inner sep=0] {
            \includegraphics[width=15cm]{Memoria/HistoriesUsuari/fitxa.png}
        };
        \draw (0, 2) node[text width=13cm, align=left] {\large{\textbf{
#1
        }}};
        \draw (0, 0.2) node[text width=13cm, align=left] {
#2
        };
        \draw (0, -2) node[text width=13cm, align=left] {\textbf{Prioritat:} 
#3
        };
    \end{tikzpicture}
}