\subsection{\textit{Timeline} i veure activitat}
\label{sec:etapa2}

Aquesta etapa té com a objectiu implementar les històries d'usuari relacionades amb el \textit{timeline} i les activitats. Com la resta de les etapes, aquesta només es pot realitzar si s'ha finalitzat la primera etapa (\nameref{sec:etapa1}). En aquesta etapa s'han d'implementar les següents funcionalitats:


\begin{compactitem}
    \item El sistema ha d'obtenir i mostrar les activitats del \textit{timeline} de l'usuari.
    \item Si l'usuari selecciona una activitat al \textit{timeline}, el sistema ha de mostrar el contingut d'aquesta. Ha de poder veure el text, la data de publicació, l'autor i els comentaris de l'activitat.
    \item L'usuari ha de poder publicar un nou comentari a una activitat.
    \item L'usuari ha de poder publicar una nova activitat, indicant el text i el context al que es publica.
    \item L'usuari ha de poder esborrar una activitat i un comentari.
    \item L'usuari ha de poder publicar una imatge. El sistema ha de mostra una pre-visualització a sota del text de l'activitat.
    \item L'aplicació ha de ser capaç d'obrir fitxers des de qualsevol altra aplicació. Quan l'usuari ho realitzi, se li ha de mostrar la vista de publicació d'una activitat, indicant que està adjuntant un fitxer.
\end{compactitem}

Aquesta etapa inclou les històries d'usuari 3, 4, 5, 6, 22, 23 i 27 (pàgines \pageref{sec:historia_3} i \pageref{sec:historia_22}).

\subsubsection{Implementació}
Per completar la primera etapa es va obtenir la informació del \textit{timeline} de l'usuari. Però aquesta informació no es mostrava amb la interfície, només es mostrava per consola. En aquesta etapa aquesta informació s'havia de mostrar seguint les directrius que havia establer el client.

A la pestanya del \textit{timeline} es va afegir una taula amb les cel·les personalitzades. Es va decidir utilitzar la taula nativa d'iOS (\textit{UITableView}) per a mostrar la informació de les activitats ja que d'aquesta manera es podien aprofitar totes les funcionalitats d'aquest element (per exemple la possibilitat d'esborrar un element lliscant amb el dit cap a un costat). En aquesta cel·la es mostra el nom de l'autor de l'activitat juntament amb la seva imatge, la data de publicació i un tros del text de l'activitat.

\pintaDosImatges
    {captura-timeline}
        {Resultat de la vista del \textit{timeline}.}
    {captura-veure-activitat}
        {Resultat de la vista de veure activitat.}

Un cop es va disposar de la vista del \textit{timeline} completada, es va continuar afegint una nova vista per a mostrar la informació d'una activitat. Aquesta vista es mostra quan l'usuari clica a una cel·la al \textit{timeline}, aleshores el sistema carrega la informació de l'activitat del servidor i la mostra a l'usuari.

En aquesta nova vista, la vista de l'activitat, es mostra el nom de l'autor de l'activitat, la data de quan es va publicar i el text complert de l'activitat. Al costat d'aquesta informació es mostra la imatge de l'autor. En cas que l'activitat disposi de comentaris, aquests es mostren a sota en format de llista. 

Per a posar els comentaris també es va optar per utilitzar una taula nativa d'iOS pels mateixos motius que al \textit{timeline}.
La funcionalitat d'afegir comentaris es va implementar amb una nova finestra que es mostra quan l'usuari clica al botó ``+'' de la barra de navegació.


Aquesta vista conté un camp de text i dos botons a la barra de navegació: ``Desar'' i ``Cancel·lar''. Si l'usuari després d'introduir un text el desa, es fa la petició al servidor per afegir el comentari. Un cop el servidor ha creat el comentari l'aplicació tanca la vista d'afegir un comentari i refresca els comentaris de la vista veure activitat.


\pintaDosImatges
    {captura-publicar-comentari}
        {Resultat de la vista de publicar comentaris.}
    {captura-publicar-activitat-buit}
        {Resultat de la vista de publicar activitat.}


Les funcionalitats esborrar activitat i esborrar comentari es van afegir utilitzant la funcionalitat de les taules natives d'iOS. Detectant l'acció de l'usuari de ``lliscar'' cap a la dreta una cel·la de la taula i mostrant un botó d'esborrar. Un cop l'usuari clica el botó, el sistema demana confirmació a l'usuari, i si l'usuari accepta, el sistema realitza la petició al servidor per a realitzar l'esborrat.

Per a implementar la confirmació que demana el sistema a l'usuari, es va utilitzar un diàleg natiu d'iOS (\textit{UIAlertView}) amb dos botons, un per acceptar i l'altre per a cancel·lar.

Per a desenvolupar la funcionalitat d'afegir noves activitats, es va partir de la d'afegir comentari. Es va afegir a sobre del camp de text la informació de l'usuari i als contexts als que es publica.

El client va demanar que per triar els contexts als que publicar, es mostrés una nova vista amb la llista de contexts als que l'usuari té permís per a publicar. En aquesta vista l'usuari havia de poder fer una selecció múltiple de contexts i tornar a la vista de nova activitat.


\pintaDosImatges
    {captura-publicar-activitat-selec-context}
        {Resultat de la vista d'afegir contexts.}
    {captura-publicar-activitat-imatge}
        {Resultat de la vista de publicar una activitat amb una imatge.}

Un cop l'usuari ha introduït el contingut de l'activitat i ha seleccionat els contexts als que publicar, el sistema fa les crides necessàries per a publicar les activitats al servidor.
Per a obrir aquesta vista l'usuari ha de clicar al botó ``+'' de la barra de navegació del \textit{timeline}.

Inicialment les activitats només havien de contenir text, però gracies a utilitzar la metodologia \textit{scrum}, el client durant el transcurs del projecte va necessitar la funcionalitat d'afegir imatges/fitxers a les activitats, i per tant es va afegir com a funcionalitat requerida. Al afegir aquesta funcionalitat es van reajustar algunes funcionalitats, simplificant-les. No va ser necessari treure cap funcionalitat.

Aquest nou requisit implicava dues tasques, l'usuari havia de poder afegir una imatge des de l'aplicació (de la galeria d'imatges o de la càmera) i l'usuari havia de poder afegir fitxers des d'una altra aplicació del dispositiu. 

La primera tasca va implicar afegir dos botons a la interfície (un per la galeria i un per la càmera), que al clicar obri el diàleg natiu corresponent. Un cop l'usuari ha triat la imatge, aquesta s'afegeixi a sota del text de l'activitat en format miniatura. A més s'han de canviar els dos botons per un d'una paperera per a poder esborrar la imatge.

Per a la segona tasca, es va modificar el fitxer de configuració de l'aplicació. Es va declarar que l'aplicació pot obrir tot tipus de fitxers. A més, es va preparar l'aplicació per a que al obrir un fitxer amb l'aplicació, se li mostri a l'usuari la vista de nova activitat. A sota del camp de text de la vista, es mostra una capsa amb el nom del fitxer que s'està adjuntant.

En tots dos casos, quan l'usuari crea la nova activitat, es fan les crides corresponents al MAX per a enviar el fitxer.


\pintaDosImatges
    {captura-obrir-fitxer}
        {Adjuntar un fitxer des d'una altra aplicació.}
    {captura-adjuntar-fitxer}
        {Resultat de la vista de publicar una activitat amb un fitxer.}
\clearpage



