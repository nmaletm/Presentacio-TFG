\subsection{Gestió de subscripcions}

L'objectiu d'aquesta etapa és el de dotar a l'aplicació la funcionalitat de gestionar les subscripcions de l'usuari. És a dir, oferir a l'usuari la possibilitat de veure les seves subscripcions, esborrar-ne una i afegir-ne de noves.

Per a considerar l'etapa com a finalitzada, l'aplicació ha de tenir les següents funcionalitats:

\begin{compactitem}
    \item El sistema ha d'obtenir i mostrar les subscripcions de l'usuari.
    \item Si l'usuari selecciona una subscripció, el sistema ha de mostrar les activitats del context seleccionat.
    \item L'usuari ha de poder buscar contexts als que es pugui subscriure.
    \item Si l'usuari ha buscat un context i el selecciona s'ha de subscriure.
    \item L'usuari ha de poder esborrar una subscripció.
\end{compactitem}

Aquesta etapa inclou les històries d'usuari 7, 8, 9, 10 i 24 (pàgines \pageref{sec:historia_7} i \pageref{sec:historia_24}).

En aquesta etapa es poden aprofitar algunes vistes definides a ``\nameref{sec:etapa2}''. Com que les etapes no impliquen un ordre, en cas que el client prioritzi aquestes funcionalitats per davant de les de l'altra etapa es realitzaran en aquesta etapa.

\subsubsection{Implementació}

El primer que es va fer va ser afegir una nova pestanya a l'aplicació i en aquesta pestanya afegir una taula amb el seu controlador. Aquesta taula obtenia les dades des les subscripcions de l'usuari i les agrupava per tipologia del context.

Per a distingir les diferents tipologies dels contexts, aquests al atribut \textit{tags} contenen una etiqueta pre-definides. A la taula \ref{tagsContexts} es poden veure les diferents etiquetes i els seus significats. Si un context no disposa de cap pre-definida, aleshores es classifica com a ``Altres''.

\begin{table}[h]
    \begin{center}
    \begin{tabular}{| l | l |}
        \hline
            \textbf{\textit{Etiqueta}}  & \textbf{Significat}           \\ 
        \hline
            {[ASSIG]}                     & Context d'una assignatura     \\ 
            {[INST]}                      & Context institucional         \\
            {[COMMUNITY]}                 & Context d'una comunitat       \\
        \hline
    \end{tabular}
    \end{center}
    \caption{ Significat de les diferents etiquetes. \label{tagsContexts}}
\end{table}

Després es va crear una nova vista similar a la del \textit{timeline} però carregant només les activitats d'un context. Aquesta vista es mostra quan l'usuari selecciona un context de la llista de subscripcions.

Gràcies a la gran similitud entre aquesta vista i la del \textit{timeline}, es va poder re-utilitzar una part de la feina feta a l'etapa ``\nameref{sec:etapa2}''. Es va utilitzar la cel·la del \textit{timeline} ja que a les dues situacions es mostrava una activitat i es necessitava mostrar la mateixa informació. Com que s'havia de tornar a mostrar la informació de l'activitat al clicar a la cel·la, també es va re-utilitzar la vista de veure una activitat.



\pintaDosImatges
    {captura-subscripcions}
        {Resultat de la vista de subscripcions.}
    {captura-subscripcio-activitats}
        {Resultat de la vista d'activitats d'un context.}


També s'havia d'oferir a l'usuari la possibilitat de publicar una activitat al context. Es va aprofitar la vista d'afegir una activitat fent una petita modificació per a poder obrir la vista amb un context ja seleccionat.

Per acabar aquesta etapa, s'havia de poder esborrar una activitat i una subscripció a les activitats d'un context i a la vista de les subscripcions respectivament. Per fer-ho es va seguir la forma en que s'ha definit anteriorment.

