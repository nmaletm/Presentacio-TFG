\subsection{Perfil}

L'objectiu d'aquesta etapa és el d'oferir a l'usuari la possibilitat de veure el seu perfil i gestionar-lo.  S'inclouen les següents funcionalitats:

\begin{compactitem}
    \item El sistema ha d'obtenir les dades de l'usuari i les seves activitats.
    \item Si l'usuari selecciona una activitat, el sistema ha de mostrar el contingut i els comentaris de l'activitat.
    \item L'usuari ha de poder modificar les seves dades personals (nom i nom d'usuari del \textit{Twitter}.
    \item L'usuari ha de poder tancar la sessió.
\end{compactitem}

Aquesta etapa inclou les històries d'usuari 11, 12, 13, 26 (pàgines \pageref{sec:historia_11} i \pageref{sec:historia_26}).

En aquesta etapa es poden aprofitar algunes vistes definides a \nameref{sec:etapa2}. Com que les etapes no impliquen un ordre, en cas que el client prioritzi aquestes funcionalitats per davant de les de l'altra etapa es realitzaran en aquesta etapa.

\subsubsection{Implementació}

La primera tasca d'aquesta etapa és la d'afegir una nova pestanya ``Pefil'' amb el seu controlador de vista.

En aquesta vista s'ha de mostrar la informació de l'usuari conjuntament amb la seva imatge. Per tant s'ha de carregar del MAX el perfil de l'usuari. Un cop està carregat s'ha de mostrar el nom, el nom d'usuari i el \textit{twitter} de l'usuari.

\pintaDosImatges
    {captura-perfil}
        {Resultat de la vista del perfil.}
    {captura-perfil-editant}
        {Resultat de la vista del perfil en mode d'edició.}
        
A continuació s'han de mostrar les activitats publicades per l'usuari. Per fer-ho, seguint el patró definit prèviament, es va afegir una taula amb les cel·les personalitzades (re-utilitzant les del \textit{timeline}). Es fa afegir la interacció de seleccionar una cel·la que obrira la vista de veure activitat.

També es va afegir la interacció per esborrar les activitats, lliscant la cel·la cap a un costat. Quan l'usuari realitza aquesta acció, abans d'esborrar l'activitat, se li demana una confirmació per evitar esborrats accidentals.

El client va demanar que es pogués modificar el perfil de l'usuari. Per a fer-ho, es va establir que al costat del nom s'ha d'afegir una icona d'un llapis. Al clicar a sobre, la vista s'ha de transformar i a on abans es mostrava el nom ara s'ha de mostrar un camp de text amb el nom. De la mateixa manera s'ha de poder canviar el \textit{twitter}.


Quan el mode d'edició està activat, a la barra de navegació s'ha de mostrar un botó de cancel·lar i un de guardar. Si l'usuari clica al primer, s'han de descartar els canvis. Però si l'usuari clica al de guardar, s'ha de fer la modificació al servidor per a modificar el perfil de l'usuari amb els canvis introduïts per l'usuari. 

Quan no està activat el mode d'edició, s'ha de mostrar un botó a la barra de navegació per a poder tancar la sessió de l'usuari. Un cop tancada la sessió, s'ha de mostrar la vista d'iniciar sessió.

\pintaDosImatges
    {captura-perfil-sortir-confirmacio}
        {Missatge de confirmació abans de tancar sessió.}
    {captura-login}
        {Resultat de la vista d'iniciar sessió.}

\clearpage
