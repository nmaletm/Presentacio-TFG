\subsection{Context}

En aquest projecte hi han moltes parts interessades involucrades, a continuació es veuran una per una quin rol tenen dintre del SomUPC i quin al projecte del treball final de grau.

\subsubsection{UPCnet}
UPCnet, al SomUPC, principalment té el rol de client, però també té el rol de desenvolupador d'algunes parts (per exemple el MAX).

Inicialment el SomUPC va sorgir del premi Davyd Luque \cite{premi_upcnet}. En concret a partir de 3 propostes: ``UPCesfer@'' (edició 2008)\footnote{\url{http://llocs.upc.edu/premidavydluque/2008/html/propostes.htm}}, ``Directori UPC'' (edició 2009)\footnote{\url{http://www.upcnet.es/premi-davydluque/edicions-anteriors/edicio-2009}} i ``U4U - University for you'' (edició 2011)\footnote{\url{http://www.upcnet.es/premi-davydluque/edicions-anteriors/2011}}. UPCnet quan estava iniciant el projecte va detectar que l'inLab FIB\cite{inLabFIB} disposava del Racó que tenia moltes similituds amb algunes parts del SomUPC.

En aquest projecte igual que al SomUPC, UPCnet té el rol de client. Com a tal, UPCnet vol obtenir una eina de col·laboració que pugui implantar a la UPC i alhora a qualsevol organització que ho pugui necessitar.

\subsubsection{inLab FIB}
Com s'ha vist en l'apartat anterior, l'inLab FIB és l'encarregat de desenvolupar una part del SomUPC, en concret el portal SomUPC, l'aplicació nativa per a Android i l'aplicació nativa per iOS.
L'equip del SomUPC dintre de l'inLab FIB està format pel director, el co-director i l'autor d'aquest projecte i dos membres més.
Al projecte, l'inLab FIB és l'encarregat de desenvolupar el producte.

Com a desenvolupadors del sistema, l'inLab FIB vol finalitzar el projecte en els terminis establerts per concloure amb èxit el projecte i mantenir una bona relació amb el client.


\subsubsection{Director i co-director}
El director i co-director d'aquest projecte són membres de l'inLab FIB. El co-director és el cap de projecte del SomUPC i el director és el cap de l'equip desenvolupador.
Al projecte tindran els mateixos rols que al SomUPC.

Com a membres de l'inLab FIB, el seu objectiu és el mateix que s'ha descrit en l'apartat anterior.

\subsubsection{Autor}
L'autor d'aquest treball al SomUPC té el rol de desenvolupador. Al projecte farà el rol d'analista, dissenyador, desenvolupador i \textit{tester}. És una de les parts més interessades, ja que el seu objectiu és entregar el treball dins del termini establert per tal de poder finalitzar el Grau que està cursant.

Com a membre de l'inLab FIB, el seu objectiu és el mateix que s'ha descrit anteriorment, a més dels objectius concrets del treball final de grau.


\subsubsection{Usuari final}
Els usuaris finals són una part interessada molt important ja que el sistema està dirigit a ells. Inicialment els usuaris finals seran els membres de la comunitat UPC, però potencialment ho pot ser qualsevol membre d'una comunitat, com pot ser una empresa.

L'objectiu dels usuaris finals és obtenir una plataforma que els hi faciliti comunicar-se amb altres membres de la seva comunitat, mitjançant la publicació d'activitats i les converses directes amb un altre usuari o un grup.

Si es vol que el projecte tingui èxit, s'ha de desenvolupar un sistema robust, còmode i usable per a que els usuaris finals estiguin contents i facin un bon ús.


