\subsection{Mètodes de validació}

Al seguir la metodologia \textit{Scrum}, no es necessita definir criteris d'acceptació de tot el projecte sencer, ja que la pròpia dinàmica ofereix un fort control del producte per part de negoci.

Un dels millors aspectes d'aquesta metodologia és la gran cohesió que crea entre tots els membres involucrats al projecte. En general sempre hi ha una mateixa visió del producte que s'està creant, i negoci pot anar perfilant el producte durant el transcurs dels \textit{sprints}.

No es defineixen criteris d'acceptació de tot el producte, però si que es defineixen criteris d'acceptació de les diferents tasques. Gràcies a que les tasques tenen una granularitat molt petita, els criteris d'acceptació poden ser molt precisos i concisos.

Com s'ha vist anteriorment, els encarregats de validar els criteris d'acceptació de les tasques són els membres de negoci, durant les reunions de demostració.