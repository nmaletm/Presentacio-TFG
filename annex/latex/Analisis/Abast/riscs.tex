\subsection{Riscs}

En aquest projecte s'han de tenir en compte els riscs involucrats al procés de desenvolupament per detectar-los el més aviat possible. Per aquest projecte s'han detectat aquests riscs:

\begin{itemize}
    \item \textbf{Producte no desitjat pel client.} Al tractar-se d'un producte destinat a un client, hi ha la possibilitat que un cop finalitzat el projecte, el client rebi un producte que no és el que esperava obtenir. Per evitar aquest risc, com s'ha indicat en els apartats anteriors, es segueix la metodologia \textit{Scrum} i per tant el client obté un \textit{feedback} continuat durant els \textit{sprints}. Si en algun moment el client detecta que el projecte no és el que espera, el client pot comunicar-ho al següent \textit{sprint} i re-orientar el projecte.
    \item \textbf{Control del temps.} Un risc molt important és excedir el termini de temps establert per a aquest projecte. Per evitar aquest risc, es farà un control exhaustiu per tal de que es compleix la planificació establerta.
    \item \textbf{Producte poc eficient.} Al tractar-se d'un producte final, destinat a tot tipus d'usuari, el producte resultant ha de tenir un bon rendiment en velocitat d'ús. Per evitar aquest risc, durant el desenvolupament del projecte es tindrà molt en compte la velocitat d'execusió.
    \item \textbf{Ús inadequat.} Com que al producte final podran accedir tot tipus d'usuari, hi ha la possibilitat que algun usuari faci un ús inadequat del producte. Per evitar aquest risc, durant el desenvolupament es tindrà en compte la seguretat de l'aplicació per evitar usos inadequats.
\end{itemize}                                    