\subsubsection{Entorn de col·laboració}

Actualment hi ha moltes eines disponibles a la xarxa per treballar en col·laboració en una comunitat. Totes permeten comunicar informació entre un grup determinat de persones. Al mercat es poden trobar les següents aplicacions que ofereixen requisits similars als que demana el client:

\paragraph{Comunicació entre usuaris\\}
Hi ha moltes plataformes que ofereixen als usuaris la possibilitat de comunicar-se. Algunes ho tenen com a propòsit principal, com és el cas de \textit{Whatsapp}\cite{whatsapp} o \textit{Hangouts}\cite{hangouts} de \textit{Google}, mentre que altres aplicacions ho tenen com a part del seu producte, com és el cas de \textit{Facebook}\cite{facebook} o \textit{Twitter}\cite{twitter}, però en aquest cas ofereixen més funcionalitats.

Per altre banda també hi han plataformes que tenen integrat un sistema de comunicació per facilitar el treball en col·laboració, com és el cas de \textit{Google Drive}\cite{g_drive} o \textit{Cacoo}\cite{cacoo}, però la part de comunicació només serveix per millorar l'experiència de l'usuari, ja que l'aplicació té un altre propòsit que no és el de la comunicació.

\paragraph{Publicar activitats\\}

Una altra funcionalitat sol·licitada és poder compartir informació amb els altres usuaris mitjançant la publicació d'activitats. Aquestes activitats inicialment només consten d'un text, però es pot ampliar a continguts multimèdia o altres tipus de fitxers.

Aquesta funcionalitat és la base de moltes xarxes socials, \textit{Twitter} per exemple ho permet fer mitjançant la publicació de \textit{tweets}. \textit{Facebook} també ho permet fer mitjançant la publicació d'estats al mur personal. \textit{Google +}\cite{g_plus} ho permet fer mitjançant la publicació de novetats amb un sistema similar al de \textit{Facebook}. D'aquesta manera un altre usuari pot veure les activitats publicades accedint al mur de l'usuari.

A més, es demana que es pugui publicar una activitat en un context concret (és a dir, per a una agrupació de persones concreta). \textit{Facebook} permet publicar activitats a un grup o a una pàgina de \textit{Facebook} i \textit{Google +} permet publicar novetats a un cercle concret. En tots dos casos només podran veure les activitats publicades aquells usuaris que formin part del grup/cercle o siguin fans de la pàgina.

Un usuari ha de poder publicar un comentari de l'activitat. Això ho permeten les tres xarxes socials, però de dues maneres diferents. \textit{Twitter} permet fer \textit{retweet}, publicar un nou \textit{tweet} com a resposta o fer una menció en un nou \textit{tweet}. 
En canvi, \textit{Facebook} i \textit{Google +} permeten publicar un comentari directament a l'activitat. Aquest últim mètode és el que s'ha demanat per aquest projecte.

\paragraph{Subscripció\\}

Els usuaris a les xarxes socials poden accedir als diferents murs dels usuaris per poder veure les novetats, però gairebé totes les xarxa socials tenen una pàgina on es mostren les activitats dels usuaris relacionats agrupats, sovint ordenats cronològicament o per rellevància.

\textit{Facebook} disposa de la secció ``Canal de notícies'' en que mostra les activitats dels amics, dels grups i de les pàgines a les que l'usuari es fan (per exemple, nous estats o noves fotografies), \textit{Google +} disposa d'una secció similar que es diu ``Novetats''. \textit{Twitter} també disposa del \textit{timeline}, en el que es poden veure tots els \textit{tweets} dels usuaris als que es segueixen.

El requisit d'aquest projecte és molt similar a aquestes implementacions, però ha de ser més genèric ja que ha de permetre crear grups d'usuaris molt fàcilment (anomenats contexts). Els usuaris s'han de poder subscriure als contexts públics.

D'aquesta manera el producte final es podrà adaptar a qualsevol organització (educativa o empresarial). Per exemple, en el cas de la UPC, cada departament podria tenir el seu context, però també els estudiants de la FIB de Grau podrien tenir un context propi.

L'objectiu és que tot aquell grup d'usuaris que hagi de compartir una activitat pugui tenir un context associat i crear una sub-comunitat (dintre la comunitat).
