\subsubsection{Innovació}

Com s'ha vist a l'apartat anterior, ja existeixen sistemes a la xarxa que permeten fer pràcticament tots els requisits del projecte. Aquests són els problemes que s'han detectat als sistemes existents:

\begin{compactitem}
    \item Hi ha moltes plataformes, però no hi ha una que englobi tots els requisits que vol el client en un mateix sistema.
    \item Per poder utilitzar aquests sistemes en una comunitat, tots els membres haurien d'estar registrats. En molts casos els usuaris ja disposarien d'un usuari en alguna de les plataformes, però potser no desitjarien barrejar l'àmbit professional/acadèmic amb el personal.
    \item Amb les eines actuals seria impossible englobar tots els usuaris d'una comunitat (com és la UPC) i fer que els usuaris poguessin cercar un usuari de la comunitat entre tots els usuaris del sistema.
    \item Algunes de les plataformes actuals només estan disponibles en un dels entorns que requereix el client (només web o només app nativa) i no en tots dos entorns.
\end{compactitem}

Durant la recerca de l'estat de l'art s'ha trobat una empresa catalana que ofereix un producte que disposa de molts dels requisits que demana el client, el producte es diu \textit{Zyncro}\cite{zyncro} de l'empresa \textit{Zyncro Tech}. L'equip ha provat la versió gratuïta del producte i les conclusions que s'han extret són les següents:

\begin{compactitem}
    \item No hi ha la possibilitat d'integrar el sistema a les plataformes web de la comunitat/empresa.
    \item Excepte a les versions \textit{Enterprise} i \textit{Business Communities} les dades estan als servidors d'aquest sistema i no als de l'empresa.
    \item Disposa d'una aplicació per dispositius mòbils, però és una única aplicació per totes les comunitats. No és possible que cada empresa tingui la seva pròpia aplicació, ni que ho integri a dintre d'una altra que ja estigui al mercat.
\end{compactitem}

Tenint en compte les conclusions extretes, el producte \textit{Zyncro} no s'ajusta als requisits del client.

Per aquest motiu aquest projecte té com objectiu desenvolupar un entorn que faciliti la col·laboració en xarxa dels membres d'una organització. Aquest treball final de grau, en concret es centra en l'app iOS d'aquest sistema.
