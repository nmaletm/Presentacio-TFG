
\subsubsection{\textit{Daily Scrum}}

Cada dia tot l'equip desenvolupador realitza una reunió de deu minuts. Aquesta reunió rep el nom de \textit{Daily Scrum}, també s'anomena \textit{Stand-up meeting} ja que aquesta reunió es realitza de peu. 

En aquesta reunió cada component de l'equip presenta un breu resum de la feina realitzada el dia anterior, i posteriorment diu les tasques que realitzarà durant el dia.

Qualsevol element bloquejant que pugui interferir amb alguna tasca del \textit{sprint}, es menciona durant la reunió i es comenta. D'aquesta manera es pot evitar o reduir al mínim les conseqüències. Un element bloquejant és un obstacle que evita que una part del treball es finalitzi.

Aquestes reunions, a més de millorar la coordinació entre els membres, és una font de motivació al generar la sensació d'avançar al veure que les tasques es van finalitzant.
