
\subsubsection{Iteracions - \textit{Sprints}}

Com s'ha vist prèviament, el desenvolupament es divideix en \textit{sprints}. L'objectiu principal d'aquesta divisió en iteracions és la d'entregar producte al client o a l'usuari final el més aviat possible. D'aquesta manera es poden detectar els problemes ràpidament i es pot refinar els requisits per a maximitzar les garanties d'èxit. Aquesta filosofia és ideal per a projectes en que s'estiguin desenvolupant productes innovadors, ja que en aquest àmbit la tecnologia avança molt ràpidament. Sovint la situació inicial canvia notablement durant el desenvolupament del projecte, i solucions que eren inviables o innecessàries a l'inici del projecte al final no ho són.

Al inici del \textit{sprint} es fa una reunió (anomenada \textit{sprint planning}), en la que participen tots els membres involucrats al projecte (negoci i equip desenvolupador). En aquesta reunió es decideixen les tasques que s'han de realitzar durant el \textit{sprint}. 

Prèviament a aquesta reunió s'han definit un conjunt de tasques, que estan al \textit{product backlog} prioritzades en funció del valor que aporten al projecte.

A la reunió d'\textit{sprint planning} s'estableixen les tasques que entren al \textit{sprint}. Aquestes es posen en comú, per a verificar que tots els membres entenen el mateix per cada una de les tasques i l'equip desenvolupador valora les tasques. 

A partir d'aquesta valoració i tenint en compte l'ordre de prioritat establert per negoci, es defineix una línia de tall per establir les tasques que entren al \textit{sprint}. Aquestes tasques es col·loquen al \textit{task-board} a la columna de \textit{Sprint backlog}.

El \textit{sprint} finalitza amb una nova reunió en la que es realitza una demostració del que s'ha fet durant les dues setmanes. A aquesta reunió assisteixen l'equip desenvolupador i negoci. Utilitzant el \textit{task-board}, per cada tasca de la columna de \textit{Done} l'equip desenvolupador demostra que s'ha realitzat i que verifica els criteris de satisfacció.

