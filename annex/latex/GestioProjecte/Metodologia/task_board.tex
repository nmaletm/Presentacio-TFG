
\subsubsection{Tauler de tasques - \textit{Task board}}

La \textit{task board} o tauler de tasques és una visió de l'estat de totes les tasques del \textit{sprint}. Aquest panell permet als membres de l'equip visualitzar l'estat del projecte d'aquell moment en un sol cop d'ull. Es pot saber quines tasques s'estan desenvolupant, qui ho està desenvolupant, veure les tasques que estan pendents i les tasques que ja s'han completat.

En aquest projecte s'utilitzen dues versions del panell, que contenen la mateixa informació. S'utilitza una pissarra física amb \textit{post-it} i una versió digital utilitzant l'eina \textit{Trello}\footnote{Eina de col·laboració per a organitzar els projectes en taules. En un cop d'ull, \textit{Trello} et diu en que s'està treballant, qui està treballant en què i que està en procés.}. A la figura \ref{fig:sprint-backlog-real} i \ref{fig:sprint-backlog} es poden veure els dos panells.

El panell està format per quatre columnes:

\begin{itemize}
    \item \textbf{\textit{Sprint backlog:}} Tasques preparades per a ser realitzades.
    \item \textbf{\textit{Problems:}} Tasques bloquejades.
    \item \textbf{\textit{Doing:}} Tasques que cada membre està desenvolupant.
    \item \textbf{\textit{Done:}} Tasques ja realitzades.
\end{itemize}


\begin{figure}[ht]
    \centering
    \includegraphics*[scale=0.2]{GestioProjecte/Metodologia/task-board.jpg}
    \caption{\textit{Sprint Backlog} físic de l'equip del SomUPC.}
    \label{fig:sprint-backlog-real}
\end{figure}

\begin{figure}[ht]
    \centering
    \includegraphics*[scale=0.6]{GestioProjecte/Metodologia/sprint-backlog.png}
    \caption{\textit{Sprint Backlog} de l'equip del SomUPC utilitzant \textit{Trello}.}
    \label{fig:sprint-backlog}
\end{figure}
\FloatBarrier 
