
\subsubsection{Història d'usuari}
Els requisits funcionals del projecte s'han definit utilitzant les històries d'usuari. Cada un d'aquests elements defineix una funcionalitat que ha de tenir el sistema. 

La història d'usuari consisteix en un parell de línies de text que descriu el requisit des del punt de vista de l'usuari. A més s'inclou un títol descriptiu de la història i un identificador únic. Aquest identificador no implica cap ordre ni importància, simplement és un codi que s'utilitza per identificar de forma unívoca la història. La història d'usuari també inclou els criteris d'acceptació expressats també des del punt de vista de l'usuari.

Es poden veure les històries d'usuari d'aquest projecte a la secció de \nameref{sec:histories_usuari}. Durant el desenvolupament, algunes històries d'usuari s'han fragmentat en tasques de menor granularitat, per a facilitar a l'equip desenvolupador la distribució d'aquestes dintre dels \textit{sprints}.

\begin{figure}[ht]
    \centering
    \includegraphics*[scale=0.60]{GestioProjecte/Metodologia/sprint-backlog-tasca.png}
    \caption{Detall d'una història d'usuari a mig desenvolupar utilitzant \textit{Trello}.}
    \label{fig:sprint-backlog-tasca}
\end{figure}

A la figura \ref{fig:sprint-backlog-tasca} es pot veure una captura d'una història d'usuari a mig desenvolupar. La captura mostra com es veu una història d'usuari amb la eina utilitzada, \textit{Trello}. Es pot veure la valoració de la història realitzada per l'equip desenvolupador (en aquest cas 6), el títol, el text i els criteris d'acceptació de la història (que en aquest cas són quatre, i un ja està fet).


\FloatBarrier 
