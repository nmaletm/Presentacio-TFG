\subsection{Metodologia}

Pel desenvolupament del SomUPC s'està utilitzant una metodologia àgil. En concret la metodologia \textit{Scrum}. Una metodologia flexible i àgil és una de les claus per a l'èxit del projecte. L'objectiu és el d'aprendre el que l'usuari i el client desitja i el que no desitgen. Es pot donar la situació en la que una funcionalitat inicialment sembli molt útil i fonamental, però després no obtingui una bona recepció per part dels usuaris als que va dirigit el producte.

\textit{Scrum} està basat en iteracions que reben el nom d'\textit{sprints}. Un \textit{sprint} està format per un conjunt de tasques que s'han de completar per acomplir els seus objectius. Normalment aquestes iteracions duren entre una setmana i un màxim de quatre setmanes. A la figura \ref{fig:scrum} es pot veure el flux general de la metodologia. A un \textit{sprint}, es desenvolupen completament una o més funcionalitats que es poden posar a producció al acabar la iteració. Els requisits durant els \textit{sprints} són intocables, només es poden modificar entre \textit{sprints}.


\begin{figure}[ht]
    \centering
    \includegraphics*[scale=0.5]{GestioProjecte/Metodologia/scrum.png}
    \caption[El procés \textit{Scrum}]{El procés \textit{Scrum} Font: wikimedia.org.}
    \label{fig:scrum}
\end{figure}
\FloatBarrier 

Les iteracions curtes permeten obtenir \textit{feedback} continuat dels usuaris i del client. Això permet que el producte canvi contínuament en funció de la reacció dels usuaris i del client. Aquesta metodologia accepta i propicia l'aparició de nous requisits, així com la modificació dels requisits existents. Aquests enriqueixen el producte i garanteixen l'èxit del projecte.

En aquest projecte no s'està seguint la metodologia al peu de la lletra, s'han fet algunes adaptacions per ajustar-la al equip. A continuació es detallen alguns conceptes importants i es detalla com s'ha aplicat \textit{Scrum} en aquest projecte.


\subsubsection{Iteracions - \textit{Sprints}}

Com s'ha vist prèviament, el desenvolupament es divideix en \textit{sprints}. L'objectiu principal d'aquesta divisió en iteracions és la d'entregar producte al client o a l'usuari final el més aviat possible. D'aquesta manera es poden detectar els problemes ràpidament i es pot refinar els requisits per a maximitzar les garanties d'èxit. Aquesta filosofia és ideal per a projectes en que s'estiguin desenvolupant productes innovadors, ja que en aquest àmbit la tecnologia avança molt ràpidament. Sovint la situació inicial canvia notablement durant el desenvolupament del projecte, i solucions que eren inviables o innecessàries a l'inici del projecte al final no ho són.

Al inici del \textit{sprint} es fa una reunió (anomenada \textit{sprint planning}), en la que participen tots els membres involucrats al projecte (negoci i equip desenvolupador). En aquesta reunió es decideixen les tasques que s'han de realitzar durant el \textit{sprint}. 

Prèviament a aquesta reunió s'han definit un conjunt de tasques, que estan al \textit{product backlog} prioritzades en funció del valor que aporten al projecte.

A la reunió d'\textit{sprint planning} s'estableixen les tasques que entren al \textit{sprint}. Aquestes es posen en comú, per a verificar que tots els membres entenen el mateix per cada una de les tasques i l'equip desenvolupador valora les tasques. 

A partir d'aquesta valoració i tenint en compte l'ordre de prioritat establert per negoci, es defineix una línia de tall per establir les tasques que entren al \textit{sprint}. Aquestes tasques es col·loquen al \textit{task-board} a la columna de \textit{Sprint backlog}.

El \textit{sprint} finalitza amb una nova reunió en la que es realitza una demostració del que s'ha fet durant les dues setmanes. A aquesta reunió assisteixen l'equip desenvolupador i negoci. Utilitzant el \textit{task-board}, per cada tasca de la columna de \textit{Done} l'equip desenvolupador demostra que s'ha realitzat i que verifica els criteris de satisfacció.




\subsubsection{\textit{Daily Scrum}}

Cada dia tot l'equip desenvolupador realitza una reunió de deu minuts. Aquesta reunió rep el nom de \textit{Daily Scrum}, també s'anomena \textit{Stand-up meeting} ja que aquesta reunió es realitza de peu. 

En aquesta reunió cada component de l'equip presenta un breu resum de la feina realitzada el dia anterior, i posteriorment diu les tasques que realitzarà durant el dia.

Qualsevol element bloquejant que pugui interferir amb alguna tasca del \textit{sprint}, es menciona durant la reunió i es comenta. D'aquesta manera es pot evitar o reduir al mínim les conseqüències. Un element bloquejant és un obstacle que evita que una part del treball es finalitzi.

Aquestes reunions, a més de millorar la coordinació entre els membres, és una font de motivació al generar la sensació d'avançar al veure que les tasques es van finalitzant.



\subsubsection{Tauler de tasques - \textit{Task board}}

La \textit{task board} o tauler de tasques és una visió de l'estat de totes les tasques del \textit{sprint}. Aquest panell permet als membres de l'equip visualitzar l'estat del projecte d'aquell moment en un sol cop d'ull. Es pot saber quines tasques s'estan desenvolupant, qui ho està desenvolupant, veure les tasques que estan pendents i les tasques que ja s'han completat.

En aquest projecte s'utilitzen dues versions del panell, que contenen la mateixa informació. S'utilitza una pissarra física amb \textit{post-it} i una versió digital utilitzant l'eina \textit{Trello}\footnote{Eina de col·laboració per a organitzar els projectes en taules. En un cop d'ull, \textit{Trello} et diu en que s'està treballant, qui està treballant en què i que està en procés.}. A la figura \ref{fig:sprint-backlog-real} i \ref{fig:sprint-backlog} es poden veure els dos panells.

El panell està format per quatre columnes:

\begin{itemize}
    \item \textbf{\textit{Sprint backlog:}} Tasques preparades per a ser realitzades.
    \item \textbf{\textit{Problems:}} Tasques bloquejades.
    \item \textbf{\textit{Doing:}} Tasques que cada membre està desenvolupant.
    \item \textbf{\textit{Done:}} Tasques ja realitzades.
\end{itemize}


\begin{figure}[ht]
    \centering
    \includegraphics*[scale=0.2]{GestioProjecte/Metodologia/task-board.jpg}
    \caption{\textit{Sprint Backlog} físic de l'equip del SomUPC.}
    \label{fig:sprint-backlog-real}
\end{figure}

\begin{figure}[ht]
    \centering
    \includegraphics*[scale=0.6]{GestioProjecte/Metodologia/sprint-backlog.png}
    \caption{\textit{Sprint Backlog} de l'equip del SomUPC utilitzant \textit{Trello}.}
    \label{fig:sprint-backlog}
\end{figure}
\FloatBarrier 



\subsubsection{Història d'usuari}
Els requisits funcionals del projecte s'han definit utilitzant les històries d'usuari. Cada un d'aquests elements defineix una funcionalitat que ha de tenir el sistema. 

La història d'usuari consisteix en un parell de línies de text que descriu el requisit des del punt de vista de l'usuari. A més s'inclou un títol descriptiu de la història i un identificador únic. Aquest identificador no implica cap ordre ni importància, simplement és un codi que s'utilitza per identificar de forma unívoca la història. La història d'usuari també inclou els criteris d'acceptació expressats també des del punt de vista de l'usuari.

Es poden veure les històries d'usuari d'aquest projecte a la secció de \nameref{sec:histories_usuari}. Durant el desenvolupament, algunes històries d'usuari s'han fragmentat en tasques de menor granularitat, per a facilitar a l'equip desenvolupador la distribució d'aquestes dintre dels \textit{sprints}.

\begin{figure}[ht]
    \centering
    \includegraphics*[scale=0.60]{GestioProjecte/Metodologia/sprint-backlog-tasca.png}
    \caption{Detall d'una història d'usuari a mig desenvolupar utilitzant \textit{Trello}.}
    \label{fig:sprint-backlog-tasca}
\end{figure}

A la figura \ref{fig:sprint-backlog-tasca} es pot veure una captura d'una història d'usuari a mig desenvolupar. La captura mostra com es veu una història d'usuari amb la eina utilitzada, \textit{Trello}. Es pot veure la valoració de la història realitzada per l'equip desenvolupador (en aquest cas 6), el títol, el text i els criteris d'acceptació de la història (que en aquest cas són quatre, i un ja està fet).


\FloatBarrier 





\subsubsection{Control de versions}

El control de versions en aquest projecte s'ha realitzat amb Git. Git és un sistema distribuït de control de versions. Cada membre de l'equip treballa amb el seu repositori privat a la seva màquina. Quan un membre vol sincronitzar el treball realitzat, envia els canvis al repositori compartit. A aquesta acció se li diu \textit{push}.

El sistema git s'encarrega de realitzar la major part de la barreja dels canvis. Com la majoria d'eines de control de versions, Git no pot barrejar dos versions de dos repositoris que han realitzat canvis a la mateixa part de codi. En aquests casos, Git demana a l'usuari que realitzi els canvis a mà.

El repositori compartit és privat, i el té allotjat el client als seus servidors. Tots els membres de l'equip hi poden accedir des de qualsevol lloc.









