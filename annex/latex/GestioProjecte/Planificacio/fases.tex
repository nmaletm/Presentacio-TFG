\subsubsection{Fases del projecte}

Per a desenvolupar aquest projecte s'està seguint la metodologia àgil \textit{Scrum}. Per aquest motiu no es poden definir unes fases amb data d'inici i data final, per tant només es pot dividir el projecte en sis fases que s'aniran repartint entre els diferents \textit{sprints} en funció del que demani negoci.


\paragraph{Estructura de l'aplicació\\}

En aquesta fase s'iniciarà el desenvolupament del projecte creant el \textit{core} de l'aplicació per poder continuar amb les altres fases.

\textbf{Requisits:}  Cap, aquesta és la primera fase del projecte \newline
\textbf{Dificultat estimada:}  Mitjana-Alta \newline
\textbf{Prioritat:}  Molt alta \newline
\textbf{Tasques:} 
\begin{compactitem}
    \item Crear i estructurar el projecte.
    \item Preparar l'aplicació per a que es pugui connectar amb el MAX.
    \item Iniciar sessió amb el sistema d'autenticació de la UPC.
\end{compactitem}


\paragraph{\textit{Timeline} i veure activitat\\}
Aquesta fase inclourà la funcionalitat a l'aplicació de veure les activitats recents i poder afegir-ne de noves.

\textbf{Requisits:}  Estructura de l'aplicació finalitzada \newline
\textbf{Dificultat estimada:}  Mitjana \newline
\textbf{Prioritat:}  Mitjana \newline
\textbf{Tasques:} 
\begin{compactitem}
    \item Obtenir les activitats del \textit{timeline}.
    \item Visualitzar el contingut d'una activitat del \textit{timeline}, el text de l'activitat i els seus comentaris.
    \item Publicar un nou comentari a una activitat.
    \item Esborrar un comentari (un comentari propi o a una activitat pròpia).
    \item Publicar una nova activitat, indicant el text i el context al que es publica.
    \item Esborrar una activitat (un activitat pròpia).
\end{compactitem}


\paragraph{Gestió de subscripcions\\}
Aquesta fase incorporarà al projecte la funcionalitat de veure els contexts als que l'usuari està subscrit i que l'usuari pugui gestionar-los.

\textbf{Requisits:}  Estructura de l'aplicació finalitzada \newline
\textbf{Dificultat estimada:}  Mitjana \newline
\textbf{Prioritat:}  Mitjana \newline
\textbf{Tasques:} 
\begin{compactitem}
    \item Mostrar contexts als que l'usuari està subscrit.
    \item Visualitzar activitats d'un dels contexts.
    \item Publicar a un dels contexts.
    \item Esborrar la subscripció a un dels contexts (si l'usuari te permís per poder-ho fer).
    \item Afegir una nova subscripció, per fer-ho l'usuari haurà de poder cercar contexts públics.
\end{compactitem}


\paragraph{Converses\\}
En aquesta fase s'incorporarà la funcionalitat de veure les converses de l'usuari, poder veure el contingut i afegir nous missatges a una conversa.

\textbf{Requisits:}  Estructura de l'aplicació finalitzada \newline
\textbf{Dificultat estimada:}  Mitjana \newline
\textbf{Prioritat:}  Alta \newline
\textbf{Tasques:} 
\begin{compactitem}
    \item Mostrar les converses de l'usuari.
    \item Visualitzar el contingut d'una conversa.
    \item Afegir un nou missatge a una conversa.
    \item Veure la informació d'una conversa (nom del grup i participants).
    \item Sortir o eliminar una conversa (en funció de si l'usuari és el creador del grup, o és una conversa de dues persones).
\end{compactitem}

\paragraph{Perfil\\}
Aquesta fase inclourà la funcionalitat de veure el perfil de l'usuari i que aquest pugui ser editat.

\textbf{Requisits:}  Estructura de l'aplicació finalitzada \newline
\textbf{Dificultat estimada:}  Baixa \newline
\textbf{Prioritat:}  Baixa \newline
\textbf{Tasques:} 
\begin{compactitem}
    \item Mostrar el perfil de l'usuari (nom, \textit{twitter} i activitats de l'usuari)
    \item Editar el nom i \textit{twitter} de l'usuari
    \item Afegir una nova activitat al perfil de l'usuari
    \item Tancar sessió
\end{compactitem}

\paragraph{Ampliació converses\\}
Aquesta fase incorporarà al projecte millores al sistema de converses. En concret aportarà temps real a la conversa i notificacions \textit{push}.

\textbf{Requisits:}  Fase de Converses finalitzada \newline
\textbf{Dificultat estimada:}  Alta \newline
\textbf{Prioritat:}  Mitjana \newline
\textbf{Tasques:} 
\begin{compactitem}
    \item Modificar les converses per rebre missatges en temps real.
    \item Rebre notificacions \textit{push} a través del sistema APNS\footnote{Apple Push Notification Service}.
\end{compactitem}