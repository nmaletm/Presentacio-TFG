\subsubsection{Recursos Materials}

El cost de recursos materials, al tractar-se d'un projecte en el que es desenvolupa una aplicació iOS, consisteix en un ordinador Mac per a poder desenvolupar. Actualment la gama més baixa que oferix \textit{Apple} és el mac mini que costa 649\euro\footnote{http://www.apple.com/es/mac-mini/}.

Per a poder provar l'aplicació en un dispositiu real, també es necessita un iPhone que actualment té un cost de 669\euro\footnote{http://www.apple.com/es/iphone/specs.html}.

A més per a poder instal·lar l'aplicació als dispositius, s'ha de tenir un compte de desenvolupador iOS que té un cost de 80\euro~anuals\footnote{https://developer.apple.com/programs/ios/}.

Al pressupost s'ha de comptabilitzar l'amortització d'aquests costos. Actualment a Espanya el coeficient d'amortització és del 26\%\cite{amortitzacio} per any. La fracció d'any que es dedicarà a treballar en aquest projecte, tenint en compte que la jornada laboral estàndard és de 8 hores diàries i, que en aquest projecte se li dediquen 2 hores diàries durant 160 dies, equival a 40 dies.

\begin{equation} 
\mbox{Amortització } = 1398 \textup{\euro} * 0.26 * 40 / 365 = 39.83 \textup{\euro}
\end{equation}

Per a poder desenvolupar s'utilitzarà software que ve inclòs en el sistema operatiu Mac (per exemple l'entorn de desenvolupament \textit{Xcode}, i software opensource (per exemple git per al control de versions).
