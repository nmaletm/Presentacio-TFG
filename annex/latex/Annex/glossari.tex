
\subsection{Glossari}


\textbf{Activitat}: al projecte SomUPC, és un conjunt d'informació que vol publicar un usuari a un context (o un grups de contexts) concrets. Pot ser un text, una imatge o un fitxer qualsevol.

\textbf{\textit{Activity Stream}} \cite{activityStream}: és una especificació per protocols de flux d'activitat, que s'utilitzen per a fer re-difusió de les activitats generades en serveis socials.

\textbf{APNS - \textit{Apple Push Notification Service}}: és un servei creat per \textit{Apple Inc}, que va ser llançat amb el iOS 3.0 el 17 de juny de 2009. El servei utilitza la tecnologia \textit{push} per enviar notificacions dels servidors d'aplicacions de tercers als dispositius \textit{iOS}, les notificacions poden incloure logotips, cançons o alarmes de text personalitzades.

\textbf{Context}: al projecte SomUPC, és un conjunt de persones que comparteixen alguna característica en comú, per exemple tots els alumnes d'una assignatura o tots els membres d'una facultat.

\textbf{\textit{Daily scrum}}: reunió diària de 10 minuts de l'equip de desenvolupament, per revisar les tasques realitzades el dia anterior i planificar les tasques a fer durant el dia.

\textbf{MVC - Model vista controlador}: patró de disseny per al desenvolupament de programari que separa el model de dades, la interfície usuari i la lògica de control.

\textbf{Negoci}: client del producte. Negoci està representat pel \textit{product owner} dintre de \textit{Scrum}.

\textbf{Notificació push}: descriu un estil de comunicacions a Internet on la petició d'una transacció s'origina en el servidor. Al contrari a la tecnologia pull, on la petició és originada en el client-servidor.

\textbf{\textit{Product owner}}: representa la veu del client. S'assegura que el resultat del treball de l'equip de Scrum s'adequa des de la perspectiva del negoci. El Product Owner escriu històries d'usuari i les prioritza.

\textbf{\textit{Scrum}}: marc de treball per a la gestió de projectes que defineix un conjunt de pràctiques, on cada persona participant assumeix un rol (\textit{Scrum master}, \textit{Product owner} i equip de desenvolupament), fet que permet adaptar-se a les necessitats i preferències de cada equip o organització.

\textbf{Scrum master}: facilitador del \textit{Scrum}, la feina principal del qual és eliminar els obstacles que impedeixen que l'equip arribi a l'objectiu de cada \textit{Sprint}.

\textbf{Sprint}: període en el que es realitza l'increment del producte, idealment aquest periode te una durada fixa. La durada fixa té com a objectiu mantenir un ritme constant i facilitar que l'abast dels requeriments associats es respecti per part del client durant la seva execució. 

\textbf{\textit{Sprint planning}}: reunió al inici de cada \textit{Sprint} on es decideix, valora i prioritza les tasques que entren al \textit{Sprint}.

\textbf{Subscripció}: acció que realitza l'usuari per rebre al seu \textit{timeline} totes les activitats d'un context.

\textbf{\textit{Timeline}}: espai on l'usuari pot veure totes les activitats que s'han publicat al sistema als contexts on està subscrit.

\textbf{Vista modal}: pantalla que es mostra per sobre de totes les pantalles actives. Si la pantalla que mostra la vista modal té una barra de navegació, la nova vista es superposa i cobreix la pantalla anterior, inclosa la barra de navegació.
