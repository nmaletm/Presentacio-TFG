\section{Informe de seguiment}


A principis de setembre el projecte es troba al \textit{sprint} 14 dels 16 que formen el projecte. Del total de 22 històries d'usuari s'han completat totes excepte la història número 22 que es finalitzarà durant aquesta iteració. A la taula \ref{historesPerSprint} podem veure les històries que s'han realitzat a cada \textit{sprint}.



    
\begin{table}[h]
    \begin{center}
    \begin{tabular}{ |c|c|c|c|c|c|c|c|c|c|c|c|c|c|c|c| }
             \hline
1    &    2	&	3	&	4	&	5	&	6	&	7	&	8	&	9	&	10	&	11	&	12	&	13	&	14	&	15	&	16
 \\ \hline
1,2	&	3,4	&	5	&	6	&	7,8	&	9,10	&	14,15	&	16,19	&	17,18	&	11,12	&	13	&	20	&	21	&	22	&	-	&	-
            \\ \hline
    \end{tabular}
    \end{center}
    \caption{ Històries d'usuari realitzades a cada \textit{sprint}. La primera fila són tots els \textit{sprint}, i a la segona fila hi han les històries d'usuari \label{historesPerSprint}}
\end{table}



Respecte la planificació inicial s'han afegit tres històries d'usuari (la 20, 21 i 22). Les dues primeres (Notificacions de converses i Converses en temps real) durant el proces inicial de planificació del projecte es van plantejar, però es va decidir que no entrarien dintre de l'abast del projecte. Finalment el client ha demanat que si formessin part de l'abast. Pel que fa la història 22 (Publicar un arxiu a una activitat) aquesta funcionalitat es nova i l'ha demanat el client durant el desenvolupament del projecte.

Afegir aquestes funcionalitats al projecte no ha implicat modificar la planificació i per tant tampoc modificar el pressupost, ja que el client ha relaxat alguns requisits d'algunes altres històries. 

Per exemple, a la història d'usuari 11 (Veure perfil usuari), només s'ha realitzat la funcionalitat que l'usuari vegi el seu perfil d'usuari i no veure el perfil de qualsevol usuari. I a la 13 (Modificar dades perfil usuari), s'ha reduït el nombre de camps que es poden modificar a només el nom de l'usuari i el seu \textit{twitter}.

Realment no s'han finalitzat totes les histories completament als \textit{sprints} indicats a la taula \ref{historesPerSprint}, ja que en molts cassos algunes funcionalitats han estat més complexes del que havíem previst inicialment i en canvi algunes altres han estat més simples del previst. 

El que s'ha fet en el cas de les histories complexes ha estat entregar a la demostració com a mínim la part fonamental de la història. I deixar per a la següent iteració els detalls menors. Per exemple al \textit{sprint} 7, en el que s'havia previst fer les històries 14 i 15 (Veure converses usuari i Veure missatges d’una conversa), realitzar la funcionalitat de mostrar missatges va ser més complicat del que s'havia pensat i es va acabar d'ajustar durant la iteració 8.

En el cas de funcionalitats més simples del previst, s'ha aprofitat el temps per avançar funcionalitats dels \textit{sprints} següents. Aquesta situació ha passat per exemple al \textit{sprint} 12 en el que s'havia d'implementar la història 20 (Notificacions de converses), aquesta funcionalitat finalment implicava menys coses a desenvolupar del que havíem valorat. La majoria de tasques les havia de realitzar el client al servidor. Per aquest motiu es van avançar tasques de la iteració 13. En concret es va començar a investigar si existien al mercat llibreries per a la tecnologia utilitzada pel MAX per a les converses en temps real.

A més el client ha demanat fer un re-disseny de l'aplicació. Però aquest canvi no ha afectat la planificació ja que es va plantejar gairebé a l'inici del projecte i l'aplicació encara tenia pocs elements del disseny inicial.

Si finalment la història 22 (Publicar un arxiu a una activitat) es finalitza a la iteració 14, el \textit{sprint} 15 es dedicarà a arreglar petits detalls i provar l'aplicació, juntament amb el 16 que ja s'havia planificat amb aquest objectiu.

